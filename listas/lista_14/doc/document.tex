\documentclass[
	12pt,				% tamanho da fonte
	openright,			% capítulos começam em pág ímpar (insere página vazia caso preciso)
	twoside,			% para impressão em recto e verso. Oposto a oneside
	openany,			% Para nao pular folhas quando um paragrafo novo começa. Oposto de Twoside e openright
	a4paper,			% tamanho do papel.
	chapter=TITLE,		% títulos de capítulos convertidos em letras maiúsculas
	section=TITLE,		% títulos de seções convertidos em letras maiúsculas
	subsection=TITLE,	% títulos de subseções convertidos em letras maiúsculas
	subsubsection=TITLE,% títulos de subsubseções convertidos em letras maiúsculas
	english,
	brazil				% o último idioma é o principal do documento
]{abntex2}
\usepackage[brazil]{babel}
\usepackage[utf8]{inputenc} % Pacote de linguas
\usepackage[normalem]{ulem}
\usepackage[T1]{fontenc}
\usepackage{lipsum}
\usepackage{cmap}
\usepackage{graphicx}
\usepackage[brazilian, hyperpageref]{backref}
\usepackage[alf]{abntex2cite} % Citações padrão ABNT
\usepackage{rotating}
\usepackage{float}
\usepackage{color}
\usepackage{listings}    
\usepackage{inconsolata}
%\usepackage[breaklinks=true]{hyperref} % Pacote para correção de problemas com sumário

\usepackage{listings}

\lstset{language=sh}

\newcommand{\colcheted}{] }
\newcommand{\colchetee}{[}

% Params
% 1 - Escala da figura
% 2 - Caminho absoluto para a figura
% 3 - Legenda da figura
\newcommand{\includeImage}[3] {

\begin{figure}[H]
 	 \centering
  		\includegraphics[scale=#1]{#2}
  	\caption{#3}
\end{figure}

}

\title{Sistemas operacionais II \\ Trabalho 2 - Servidores FTP}
\date{\today}
\autor{Akira Kotsugai \\ Felipe Menino Carlos \\ Weslei Luiz}

\setlength{\parindent}{1.3cm}
\frenchspacing

% Adicionando idioma
\selectlanguage{brazil}

\begin{document}
\maketitle

\newpage
\tableofcontents
\newpage
\listoffigures
% \newpage
% \listoftables


\chapter{Introdução}

Geralmente usa-se um servidor FTP como servidor de arquivos, apenas para \textit{download} de arquivos. Mas também é possível também realizar \textit{upload} de arquivos. Além disso uma série de restrições de acesso e permissões podem ser estabelecidas. Você irá configurar um \textit{cluster} de servidores FTP de acordo com as seguintes especificações:

\begin{itemize}
	\item Serão dois servidores, um será um FTP anônimo, usado apenas para \textit{download} de arquivos. Para o outro será necessário autenticação por sftp, e deverá ser possível realizar o \textit{upload} de arquivos;
	\item Cada máquina deverá ter um usuário \textbf{adminFtp}. Será no diretório \textbf{home} deste usuário o armazenamento dos arquivos disponibilizados pelo servidor FTP. O grupo deste usuário deverá ser o \textbf{ftpUsers}, e para o servidor não anônimo somente os usuários pertencentes a esse grupo poderão realizar o \textit{download}/\textit{upload de arquivos};
	\item Deverá ser possível acessar os servidores tanto por um navegador como também pela linha de comando;
	\item O cluster,deverá estar na mesma rede da máquina hospedeira (Hospedeira da VM);
	\item A cota total para a pasta \textbf{home} do usuário \textbf{adminFTP}, nos dois servidores deverá ser de 1GB. 
\end{itemize}

\chapter{Requisitos}

Neste capítulo é feito o levantamento de requisitos que serão necessários para a conclusão da atividade.

\begin{itemize}
	\item Deverão haver dois servidores;
	\item O FTP anônimo deverá ser usado apenas para \textit{download} de arquivos; 
	\item O FTP por autenticação sftp, deverá ser usado para realizar o \textit{upload} de arquivos, além do \textit{download};
	\item Os dois servidores deverão ter o usuário \textbf{adminFtp};
	\item Os arquivos disponibilizados pelo servidor FTP deverão ficar no diretório home do usuário adminFtp
	\item O grupo do usuário \textbf{adminFtp} deverá ser o \textbf{ftpUsers};
	\item Para o servidor não anônimo somente os usuários pertencentes ao grupo ftpUsers poderão realizar o \textit{download}/\textit{upload} de arquivos;
	\item Deverá ser possível acessar os servidores tanto por um navegador como também pela linha de comando;
	\item O cluster, deverá estar na mesma rede da máquina hospedeira (Máquina onde estão alocadas as VMs); 
	\item A cota total para a pasta home do usuário \textbf{adminFTP}, nos dois servidores deverá ser de 1GB.
\end{itemize}

\chapter{Desenvolvimento}

Nesta capítulo será demonstrado os passos para a configuração e testes do cluster de FTPs.

\section{Topologia}

A topologia aderida para este projeto, é exibida na figura abaixo.

\includeImage{0.5}{../imgs/topologia.png}{Topologia da rede}

\section{Instalação do sistema operacional}

A instalação do sistema operacional Debian, foi realizada utilizando a menor quantidade de dependências possíveis. Foi realizado também a partição do disco, essa que pode ter sua estrutura vista abaixo:

\includeImage{0.5}{../imgs/particionamento.png}{Particionamento da máquina}

Essas foram as configurações realizadas no momento da instalação do sistema operacional Debian.

\section{Configuração dos usuários e grupos}

Para atender o requisito da existência do usuário \textbf{adminftp} em todas as máquinas, bem como o grupo \textbf{ftpusers}, veja abaixo as imagens dos comandos que foram utilizados, isso nas duas máquinas.

\includeImage{0.5}{../imgs/2_configuracao_usuarios/1_adduser_adminftp.png}{Adicionando usuario adminftp}

\includeImage{0.5}{../imgs/2_configuracao_usuarios/2-addgroup_ftpusers.png}{Adicionando grupo ftpusers}

Após as duas adições feitas acima, o usuário \textbf{adminftp} foi adicionado ao grupo \textbf{ftpusers}, e transformado em administrador do mesmo grupo.

\includeImage{0.5}{../imgs/2_configuracao_usuarios/3-gpasswd_a_adminftp_ftpusers.png}{Adicionando adminftp ao grupo ftpusers}

\section{Definição dos sistemas de quota}

Todas as configurações de quota dependem do pacote \textbf{quota}, é possível realizar a instalação do pacote quota com o comando:

\begin{lstlisting}
# apt-get install quota
\end{lstlisting}


Depois de instalado o pacote deve ser configurado, para isso, alterações no arquivo \textbf{fstab} deverão ser feitas.
O arquivo \textbf{fstab} está localizado em /etc/fstab e precisa ser alterado com a identificação \textbf{grpquota} para que o pacote quota saiba onde serão aplicas as restrições definidas.

Porém antes da alteração do arquivo /etc/fstab, será necessário parar os serviços de quota que estão sendo executados, e isso pode ser feito com o comando listado abaixo:

\begin{lstlisting}
# quotaoff -augv
\end{lstlisting}


Com o serviço desativado, faça a adição da identificação \textbf{grpquota}.

\includeImage{0.5}{../imgs/3_quota/0_flag_quota.png}{Adição das identificações de quota}

Neste caso, a partição selecionada para aplicação das regras é a \textbf{/home}

Após a configuração, será necessário verificar a integridade das configurações feitas e em seguida inicializar o serviço, veja os comandos

\begin{lstlisting}
# mount /home -o remount,quota

# quotacheck -augv

# quotaon -augv
\end{lstlisting}

Caso a execução dos comandos acima não apresentem problemas a regra de quota pode ser definida para os usuários e grupos, veja abaixo as configurações que foram aplicadas aos usuários do grupo \textbf{ftpusers}

\begin{lstlisting}
# edquota -g ftpusers
\end{lstlisting}

\includeImage{0.5}{../imgs/3_quota/3_edquota_adminftp.png}{Definindo regra de quota para o grupo ftpusers}

Ao finalizar as configurações, basta desabilitar e habilitar o serviço de quota novamente, para isso, os comandos listados abaixo foram utilizados:

\begin{lstlisting}
# quotaoff -augv

# quotacheck -augv

# quotaon -augv
\end{lstlisting}


Observação.: Em caso de dúvidas sobre o pacote quota é possível consultar o capítulo 2.5 da documentação do projeto \href{https://github.com/M3nin0/turbo-sniffle/blob/master/document.pdf}{\textbf{Cluster de máquinas}}

\section{Servidores FTP}

Neste capítulo as configurações dos dois servidores serão abordadas

\subsection{Anônimo}

O servidor FTP sem autenticação com apenas opção de \textit{download} utiliza apenas o pacote \textbf{vsftpd}, este pacote pode ser instalado com o comando \textbf{apt-get install vsftpd}.

O arquivo de configuração do serviço de FTP fica localizado no diretório /etc, no ambiente deste projeto, o arquivo fica especificamente em /etc/vsftpd.conf, porém, existem casos em que o arquivo se localiza no diretório /etc/ftp/vsftpd.conf.

As opções alteradas neste arquivo de configuração, e os respectivos valores que essas receberam, são listadas abaixo:

\begin{itemize}
	\item listen=YES
	\item listen\char`_ipv6=NO
	\item anonymous\char`_enable=YES
	\item anon\char`_root=/home/adminftp
	\item local\char`_enable=YES
	\item dirmessage\char`_enable=YES
	\item use\char`_localtime=YES
	\item xferlog\char`_enable=YES
	\item connect\char`_from\char`_port\char`_20=YES
	\item xferlog\char`_file=/var/log/vsftpd.log
	\item ascii\char`_download\char`_enable=YES
	\item ftpd\char`_banner=Server 1 FTP without authtentication.
	\item pam\char`_service\char`_name=vsftpd
	\item rsa\char`_cert\char`_file=/etc/ssl/certs/ssl-cert-snakeoil.pem
	\item rsa\char`_private\char`_key\char`_file=/etc/ssl/private/ssl-cert-snakeoil.key
	\item ssl\char`_enable=NO
\end{itemize}

Após encontrar ou digitar estas opções basta reiniciar o serviço com os comandos:

\begin{lstlisting}
# service vsftpd stop

# service vsftpd start
\end{lstlisting}


\subsection{Autênticado com conexão SSH (SFTP)}

As configurações iniciais, citadas na seção anterior, também se aplicam ao serviço de FTP autênticado, o ponto é que neste servidor também será utilizado um serviço de criptografia, para que a conexão seja segura e garanta a confiabilidade e integridade dos dados no momento da transferência.

De inicio, instale o serviço SSH na máquina que será utilizado como forma de conexão SFTP.

Para começar a configuração deste serviço, será necessário acessar o arquivo \textbf{/etc/ssh/sshd\char`_config}, e alterar os campos listados abaixo.

\begin{itemize}
	\item Subsystem sftp internal-sftp
	\item Match group *
	\item ChrootDirectory /home/adminftp
	\item X11Forwarding no
	\item AllowTcpForwarding no
	\item ForceCommand internal-sftp
\end{itemize}

Quando esses parâmetros estiverem inseridos no arquivo de configuração, basta reiniciar o serviço com os comandos:

\begin{lstlisting}
# service sshd stop

# service sshd start
\end{lstlisting}

\includeImage{0.5}{../imgs/5_sftp/arquivo_de_configuracao_sshd.png}{Arquivo de configuração sshd}

Após configurar o sshd, é necessário configurar o diretório o qual será a raiz do \textbf{sftp}. Para este caso, é necessário tornar o usuário root o \textit{owner} do diretório e aplicar a permissão 750, para isso os comandos abaixo:

\begin{lstlisting}
# chown root:ftpusers /home/adminftp
# chmod 750 /home/adminftp
\end{lstlisting}

Perceba que, neste ponto o usuário já está pronto para ser utilizado, porém com as permissões definidas no diretório raiz os usuários que realizarem a conexão conseguirão realizar o download, porém o upload ficará bloquado, sendo assim, deve-se criar alguns diretórios para que os usuários realizem estas ações.

% Akira ira rever esta parte
% Realizamos a revisão junto ao professor destes requisitos
Para ter controle sobre os \textit{uploads} foi criado um diretório nomeado \textbf{upload} com as permissões 775, seu grupo é o ftpusers e o \textit{owner} para root. Um diretório \textbf{download} também foi criado com as permissões 755.

% Pode ser que não possa ser feito usando dois diretórios
\includeImage{0.5}{../imgs/5_sftp/permissoes_download_upload.png}{Permissoes diretórios download/upload}

Quando é realizado algum \textit{upload} de arquivo as permissões deste restringem sua escrita somente ao usuário que o colocou no servidor.

\includeImage{0.5}{../imgs/5_sftp/permissoes_upload_arquivo.png}{Permissões de arquivos após upload no servidor SFTP}

Como forma de solução deste problema é criado uma rotina cron com o usuário root. Nesta o grupo do diretório é alterado para \textbf{ftpusers} e o \textit{owner} para root. As permissões são 770. 
Outra rotina também é criada para disponibilizar os arquivos do diretório \textbf{upload} no diretório \textbf{download}.

\includeImage{0.5}{../imgs/5_sftp/rotina_crontab.png}{Rotina cron}

\chapter{Testes}

Para assegurar de que os requisitos levantados pelo exercício sejam concluídos, será realizado os testes das configurações e permissões de usuários/grupos e serviços.

\section{Quota}

As cotas definidas \textit{soft}, tem tamanho de 500MB, e a \textit{hard} 1GB, veja os testes para certificar de que os valores das quotas estão sendo respeitadas.

No primeiro teste é verificado que, caso o arquivo tenha tamanho maior que o limite \textbf{hard}, um erro é exibido, e a gravação é abortada, veja:

\includeImage{0.5}{../imgs/3_quota/1_test.png}{Teste de quota - Arquivo maior que o limite} 

O segundo teste mostra que, se somente o limite \textbf{soft} for exedido, apenas uma mensagem é exibida ao usuário, porém  o arquivo é gravado no disco por inteiro.

\includeImage{0.5}{../imgs/3_quota/2_test.png}{Teste de quota - Arquivo menor que o limite}

\section{Servidores FTP}

Nesta seção serão realizados os testes dos servidores FTP anônimo e autênticado

Para realizar os testes de conectividade de ambos os servidores, a ferramenta \textbf{netstat} do pacote \textbf{net-tools}, foi utilizada, para verificar se no servidor que está sendo verificado, o serviço FTP, está habilitado, e caso esteja, em qual porta o serviço está disponível, como pode ser visto na figura abaixo:

\includeImage{0.5}{../imgs/4_ftp_sem_autenticacao/1_test.png}{Verificação de funcionamento do serviço}

Além disso, ao decorrer dos testes, será necessário utilizar um cliente FTP, para isso será realizada a instalação do cliente com o seguinte comando:

\# apt-get install ftp

Após isso, o ambiente pode ser considerado configurado para processeguir com os testes.

\subsubsection{Anônimo}

Para os testes do FTP anônimo, será realizado o acesso ao servidor via \textit{browser}, e a tentativa de leitura e escrita via linha de comando no servidor.

A primeira etapa, será adquirir o endereço IP da máquina onde o servidor anônimo foi configurado, para isso, utilize o comando \textbf{ip a} na máquina.

\includeImage{0.5}{../imgs/4_ftp_sem_autenticacao/2_test.png}{Endereço IP do servidor FTP anônimo}

Com o endereço obtido, o primeiro teste será realizado no navegador de outra máquina, que está na mesma rede que a máquina virtual

\includeImage{0.5}{../imgs/4_ftp_sem_autenticacao/3_test.png}{Acesso ao FTP anônimo utilizando o navegador}

Veja que o arquivo que está sendo listado, é o mesmo presente na home do \textbf{adminftp}

\includeImage{0.5}{../imgs/4_ftp_sem_autenticacao/4_test.png}{Arquivos no diretório do adminftp}

Por fim, veja uma interação utilizando a linha de comando, nessa é realizando o \textit{download} e a tentativa de \textit{upload}

\includeImage{0.5}{../imgs/4_ftp_sem_autenticacao/5_test.png}{Interação com o FTP via linha de comando}

\subsubsection{Autênticado com conexão SSH (SFTP)}

\end{document}
