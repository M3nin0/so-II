\documentclass[
	12pt,				% tamanho da fonte
	%openright,			% capítulos começam em pág ímpar (insere página vazia caso preciso)
	%twoside,			% para impressão em recto e verso. Oposto a oneside
	openany,			%Para nao pular folhas quando um paragrafo novo começa. Oposto de Twoside e openright
	a4paper,			% tamanho do papel.
	chapter=TITLE,		% títulos de capítulos convertidos em letras maiúsculas
	section=TITLE,		% títulos de seções convertidos em letras maiúsculas
	%subsection=TITLE,	% títulos de subseções convertidos em letras maiúsculas
	%subsubsection=TITLE,% títulos de subsubseções convertidos em letras maiúsculas
	english,
	brazil				% o último idioma é o principal do documento
]{abntex2}
\usepackage[brazil]{babel}
\usepackage[utf8]{inputenc} %Pacote de linguas
\usepackage[normalem]{ulem}
\usepackage[T1]{fontenc}
\usepackage{lipsum}
\usepackage{cmap}
\usepackage{graphicx}
\usepackage[brazilian,hyperpageref]{backref}
\usepackage[alf]{abntex2cite} % Citações padrão ABNT
\usepackage{rotating}
\usepackage{float}
\usepackage{color}
\usepackage{listings}    
\usepackage{inconsolata}

\usepackage{listings}

\newcommand{\imagem}[3]{
	\begin{figure}[htb]
		\begin{center}
			\includegraphics[scale=0.5]{#1}
		\end{center}
		\caption{#2}	%\label{#3}
	\end{figure}
}

\title{Sistemas operacionais II \\ Lista de exercícios 13}
\date{\today}
\autor{Felipe Menino Carlos}

\setlength{\parindent}{1.3cm}
\frenchspacing

\lstset{language=sh}

% Adicionando idioma
\selectlanguage{brazil}

\begin{document}
\maketitle

\chapter{Introdução}

Geralmente usa-se um servidor FTP como servidorde arquivos, apenas para download de arquivos.Mas também é possível também realizar upload de arquivos. Além disso uma série de restrições de acesso e permissões podem ser estabelecidas. Você irá configurar um cluster de servidores FTP de acordo com as seguintes especificações:
\begin{itemize}
	\item Serão dois servidores, um será um FTP anônimo, usado apenas para download de arquivos.Para o outro será necessário autenticação por sftp, e deverá ser possível realizar o upload de arquivos;
	\item Cada máquina deverá ter um usuário \textbf{adminFtp}. Será no diretório \textbf{home} deste usuário o armazenamento dos arquivos disponibilizados pelo servidor FTP. O grupo deste usuário deverá ser o \textbf{ftpUsers}, e para o servidor não anônimo somente os usuários pertencentes a esse grupo poderão realizar o download/upload de arquivos;
	\item Deverá ser possível acessar os servidores tanto por um navegador web (ex:ftp://192.168.1.1) como também pela linha de comando;
	\item O cluster,deverá estar na mesma rede da máquina hospedeira (Hospedeira da VM);
	\item A cota total para a pasta \textbf{home} do usuário \textbf{adminFTP}, nos dois servidores deverá ser de 1GB. 
\end{itemize}


\chapter{Levantamento de requisitos}

\end{document}