\documentclass[
	12pt,				% tamanho da fonte
	%openright,			% capítulos começam em pág ímpar (insere página vazia caso preciso)
	%twoside,			% para impressão em recto e verso. Oposto a oneside
	openany,			%Para nao pular folhas quando um paragrafo novo começa. Oposto de Twoside e openright
	a4paper,			% tamanho do papel.
	chapter=TITLE,		% títulos de capítulos convertidos em letras maiúsculas
	section=TITLE,		% títulos de seções convertidos em letras maiúsculas
	%subsection=TITLE,	% títulos de subseções convertidos em letras maiúsculas
	%subsubsection=TITLE,% títulos de subsubseções convertidos em letras maiúsculas
	english,
	brazil				% o último idioma é o principal do documento
]{abntex2}
\usepackage[brazil]{babel}
\usepackage[utf8]{inputenc} %Pacote de linguas
\usepackage[normalem]{ulem}
\usepackage[T1]{fontenc}
\usepackage{lipsum}
\usepackage{cmap}
\usepackage{graphicx}
\usepackage[brazilian,hyperpageref]{backref}
\usepackage[alf]{abntex2cite} % Citações padrão ABNT
\usepackage{rotating}
\usepackage{float}
\usepackage{color}
\usepackage{listings}    
\usepackage{inconsolata}

\usepackage{listings}

\newcommand{\imagem}[3]{
	\begin{figure}[htb]
		\begin{center}
			\includegraphics[scale=0.5]{#1}
		\end{center}
		\caption{#2}	%\label{#3}
	\end{figure}
}

\title{Sistemas operacionais II \\ Lista de exercícios 13}
\date{\today}
\autor{Felipe Menino Carlos}

\setlength{\parindent}{1.3cm}
\frenchspacing

\lstset{language=sh}

% Adicionando idioma
\selectlanguage{brazil}

\begin{document}
\maketitle

\chapter{Exercícios}

Lista com exercícios de configurações básicas de redes.

\section{Exercício 1}

Listar as informações dos arquivos /etc/hostname, /etc/hosts e/etc/network/interfaces

Realizando \textbf{login} com o usuário \textbr{root}.
\begin{lstlisting}
su
\end{lstlisting}

Listando as informações, como demonstra a figura abaixo.

\imagem{exe_1.png}{Informações de hostname, hosts e interfaces}

\section{Exercício 2}

Alterar as informações do arquivo /etc/network/interfaces para:

\begin{lstlisting}
auto lo
iface lo inet loopback
auto eth0
iface eth0 inet static
    address 192.168.200.x
    netmask 255.255.255.0
    network 192.168.200.0
    broadcast 192.168.200.255
    gateway 192.168.200.254
\end{lstlisting}

Para realizar esta configuração, será acessado o \textbf{/etc/network/interfaces}

\begin{lstlisting}
vim /etc/network/interfaces
\end{lstlisting}

A configuração pode ser vista na figura abaixo:

\imagem{exe_2.png}{Configuração de rede}

Veja que neste caso o endereço utilizado foi diferente do demonstrado no exercício,pois a rede utilizada para teste também é diferente.

Após configurar, será necessário reinicializar a interface de rede, utilizando:

\begin{lstlisting}
/etc/init.d/networking restart
\end{lstlisting}

\section{Exercício 3}

Listar o tráfego de rede com o comando netstat.

Para utilizar o comando netstat, será necessário instalar o pacote \textbf{net-tools}, veja:

\begin{lstlisting}
apt install net-tools
\end{lstlisting}

Com a instalação feita, basta executar o comando \textbf{netstat -tan}

\imagem{exe_3.png}{Imagem do trafego de dados}

\end{document}
